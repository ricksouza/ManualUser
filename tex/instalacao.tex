\section{Introdução}

A instalação do sistema de emissão de certificados ICPEdu é feita de forma automatizada e todos os requisitos já são instalados junto com ele, quando executados os comandos necessários.

\section{Pré-instalação}

Prepare uma máquina com uma instalação Ubuntu 14.04 32 bits LTS. Abra o site do projeto (\textit{https://projetos.labsec.ufsc.br/saec}) e entre na página "Lançamento da versão 1.0.0", pois todos os arquivos que deverão ser baixados estão disponíveis nesse endereço.

\section{Passo a passo}

\subsection{Instalação via repositório apt}

\begin{enumerate}
  \item Primeiramente baixe a chave pública PGP do repositório, utilizando o comando:  
  
    \textit{\$ gpg --keyserver keyserver.ubuntu.com --recv-keys 0AC9A8D0} \\
    
    A seguinte mensagem deve aparecer durante a execução do comando:\\
    gpg: Requisitando chave 0AC9A8D0 de servidor hkp - keyserver.ubuntu.com\\
    gpg: Chave 0AC9A8D0: chave pública "LabSec APT Repository $<labsec-l@inf.ufsc.br>$" importada\\
    gpg: Ultimamente não encontradas chaves confiáveis\\
    gpg: Número total processado: 1\\
    gpg:              importados: 1  (RSA: 1)
    
  \item Em seguida, adicione a chave do chaveiro do apt com o comando:  
  
  \textit{\$ gpg --armor --export 0AC9A8D0 | sudo apt-key add -}
  
  \item Baixe o script configureRepository.sh e execute:  
  
  \textit{\$ sudo chmod +x configureRepository.sh}\\
 \textit{\$ sudo ./configureRepository.sh}
  
  \item Atualize a lista de pacotes do apt:  
  
  \textit{\$ sudo apt-get update}
  
  \item E instale o sistema:  
  
  \textit{\$ sudo apt-get install saec}
  
  \item Para acessar, basta abrir um navegador de sua escolha e digitar: 
  
  \textit{https://ipdasuamaquina:40444}.
\end{enumerate}

\subsection{Instalação a partir do código fonte}

\begin{enumerate}
  \item Primeiramente baixe o sistema de emissão de certificados ICPEdu e extraia os arquivos. Entre no diretório onde está a pasta descompactada e dê permissão de execução no arquivo \textit{wgets.sh}:  
  
    \textit{\$ cd diretório/} \\
    \textit{\$ chmod +x scripts/wgets.sh} \\
    \textit{\$ ./scripts/wgets.sh}
    
  \item Baixe e instale os arquivos \textit{ LibCryptoSEC} e \textit{PHP5-LibCryptoSEC}, disponíveis no site do projeto.
  
  \item Instale o php5 e seus módulos e reinicie o apache:  
  
    \textit{\$ sudo apt-get install postgresql} \\
    \textit{\$ sudo apt-get install php5 php5-sqlite php5-ldap php5-mcrypt php5-dev php5-pgsql} \\
    \textit{\$ sudo service apache2 restart}
  
  \item Instale o banco de dados:  
  
    \textit{\$ sudo apt-get install pgadmin3} \\
    \textit{\$ sudo -u postgres psql} \\
    \textit{\$ sudo passwd postgres}
  
  \item Digite e redigite sua senha. Feito isso, entre com o usuário postgres e altere a senha para conectar ao banco:  
  
    \textit{\$ su postgres}\\
    \textit{\$ psql -c "ALTER USER postgres WITH PASSWORD '1234'" -d template1}
    
  \item Para criar tabelas e popular o banco de dados, abra o pgadmin3 e clique no botão \textit{add conection to a server}. Peencha os campos com as seguintes informações: 
  
  Name: saec\\
  Host: localhost\\
  Port: 5432\\
  Username: postgres\\
  Password: 1234*
  
  \item Clique com o botão direito em \textit{Databases}, \textit{New Database}. Preencha o campo:
  
  Name: saec\_dev
  
  \item Clique no botão \textit{Execute arbitrary SQL queries} (ao lado do botão que se parece com uma lixeira) e execute os arquivos \textit{createDatabase.sql} e \textit{populateDatabase.sql}, seguindo o caminho:
  
    diretórioRaizDoProjeto/application/modules/db/sql/createDatabase.sql \\
    diretórioRaizDoProjeto/application/modules/db/sql/populateDatabase.sql
  
  \item Instale e habilite o módulo do Shibboleth:
  
    \textit{\$ sudo apt-get install libapache2-mod-shib2} \\
    \textit{\$ sudo a2enmod ssl} \\
    \textit{\$ sudo a2enmod shib2}\\
    \textit{\$ sudo a2enmod rewrite}\\
    \textit{\$ sudo apt-get install libp11-dev}\\
    \textit{\$ sudo php5enmod mcrypt}
    
    \item Gere a chave e o certificado ssl:
    
      \textit{\$ openssl req -out saec.csr -new -newkey rsa:2048** -nodes -keyout saec.key} \\
      \textit{\$ openssl req -x509 -nodes -days 365** -newkey rsa:2048** -keyout saec.key -out saec.crt} \\
      \textit{\$ sudo mv saec.key /etc/shibboleth/}\\
      \textit{\$ sudo mv saec.crt /etc/ssl/certs/}
      
    \item Crie o Virtual Host:
    
      \textit{\$ sudo mv diretorioraizdoprojeto/\_util/packages\_generator/SAEC/etc/apache2/sites-available/saec\_default /etc/apache2/sites-available/saec.conf}
    
    \item Abra com algum editor de texto e modifique as seguintes configurações:
    
    DocumentRoot "caminhoAbsolutoParaODiretórioRaizDoProjeto/diretórioRaizDoProjeto/public" \\
    SSLCertificateFile /etc/ssl/certs/saec.crt \\
    SSLCertificateKeyFile /etc/shibboleth/saec.key \\
    DocumentRoot "caminhoAbsolutoParaODiretórioRaizDoProjeto/diretórioRaizDoProjeto/public" \\
    Directory "caminhoAbsolutoParaODiretórioRaizDoProjeto/diretórioRaizDoProjeto/public" \\
    Directory "caminhoAbsolutoParaODiretórioRaizDoProjeto/diretórioRaizDoProjeto/public/signup" 
    
    \item Crie uma pasta para os logs do sistema:
    
    \textit{\$ sudo mkdir /var/log/apache2/saec}
    
    \item Reinicie o apache:
    
    \textit{\$ sudo service apache2 restart}\\
    \textit{\$ sudo a2ensite saec}\\
    \textit{\$ sudo service apache2 reload}
    
    \item Dê permissão total de acesso para o seu diretório do projeto:
    
    \textit{\$ sudo chmod -R 777 ~/diretorio}
    
  \item Para acessar o sistema de emissão de certificados ICPEdu, basta abrir um navegador de sua escolha e digitar: 
  
  \textit{https://ipdasuamaquina:40444}.
\end{enumerate}
    \subsubsection{Notas}
    
    \begin{itemize}
  \item \textbf{Senha padrão}: A senha padrão "1234" é sugerida pois já está salva no banco de dados. Caso seja escolhida uma senha diferente, a mesma deverá ser alterada no banco de dados posteriormente;
  \item \textbf{Atributos padrão}: O tamanho da chave e o número de dias são sugestões, podendo ser alterados.
\end{itemize}
    